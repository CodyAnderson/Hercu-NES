\documentclass[a4paper]{article}
\usepackage{float}
\usepackage[english]{babel}
\usepackage[utf8x]{inputenc}
\usepackage{amsmath}
\usepackage{graphicx}
\usepackage{url}
\usepackage{gensymb}
\usepackage[colorinlistoftodos]{todonotes}
\usepackage{fancyvrb}
\usepackage[final]{pdfpages}

\title{Project Nock-off Entertainment System}
\author{Cody Anderson\\Ben Nollan\\Morgan Skrabut\\Ryan Price}
\date{\parbox{\linewidth}{\centering%
  \today\endgraf\bigskip
  Jeremy Thomas \endgraf\medskip
  Dept.\ of Electrical Computer Engineering \endgraf
  %DigiPen Institute of Technology \endgraf
  \centering{\includegraphics[width=.65\textwidth]{DigiPenLogo.png}}\endgraf
  ECE 310L, ECE410L, Conjoined \(3^{rd}\) and \(4^{th}\) Year CE Project }}

\begin{document}
\begin{titlepage}
	\maketitle
\end{titlepage}
\begin{abstract}
Project Nock-off Entertainment System (NoES) aims to create a Nintendo Entertainment System (NES) clone in SystemVerilog, which will be instantiated on an FPGA. The design of the NES clone will be nearly identical to the design of the actual NES console. In order to create a very similar clone, the discrete IC's on the NES will be modeled by discrete modules in SystemVerilog. Extensive validation and testing will go into checking the correctness of the timing and results of CPU and Picture Processing Unit (PPU) instructions. If we succeed in creating this clone, the NoES will be capable of running original NES games. 
\end{abstract}

\section{Introduction}
% Clearly state your engineering objective, and how you define success in terms of what your device will do.  Motivate why you want to do this project and provide background information. This should include discussions of your previous work if any, and a brief review of similar work by others in peer reviewed articles.
Remembered and loved by many individuals today, the Nintendo Entertainment System released in North America in 1985 is a gaming console icon of nostalgia and joyful memories. This console has been brought back to life by few individuals on the FPGA and we propose to achieve the same endeavor this year. We are motivated to recreate the NES ourselves because it is a very rare and highly sought after console with its roots deep in the hearts of many gamers including ours and those of our friends, families, and academic colleagues. We will be referencing two individuals who completed this project themselves (Dan Strother \cite{strother} and Jonathan Ganyer \cite{ganyer}). Our success in completing this project will be defined by our ability to run the game Super Mario Bros. on our FPGA embedded console by the end of the Fall semester.

\section{Methods, Techniques, and Design}
% Describe how you will design and or build your device. Cite references that you consulted on relevant technologies. This should include at least one diagram.
\subsection{Block Diagram}
\begin{figure}[H]
\centering
\includegraphics[width=\textwidth]{Capture.PNG}
\caption{\label{fig:Schedule}Project Block Diagram showing simplified flow of data and physical structure of components and modules}
\end{figure}

\subsection{Central Processing Unit (CPU)}
The CPU in the NES is based on the 6502 CPU, but with a few changes. The CPU operates at a frequency of 1.79 MHz and the Audio Processing Unit (APU) is built into the same package. In terms of I/O, the CPU has a 16-bit address bus and a bi-directional 8-bit data bus.

\subsection{Picture Processing Unit (PPU)}
The NES PPU is used for generating two-dimensional scenes by using composite video output. It operates at 21.48 Mhz and outputs one pixel per clock cycle. It also has a dedicated 2K SRAM Chip for storing data. This chip communicates with the CPU by using the CPU data bus. It also is connected to three of the CPU address lines, these are used to select the PPU register.

\subsection{Chip Interconnects}
In the NES, all of the devices share the data lines and take turns outputting to them. This method won't work when using SystemVerilog to describe the hardware as it doesn't have the capability to resolve multiple driving nets. The current solution that we intend to use is the method of separating the inputting and outputting of each device. Then all the data buses will be passed through a multiplexer. This system should mitigate some of the issues with multi driver nets and shouldn't pose any problems with data propagation.

\subsection{Hardware Interface}
We are going to connect a cartridge interface and two controllers to the FPGA using the two 40-pin expansion ports on the FPGA. The cartridge is a 72-pin socket, most of which are used. Each controller connector has 7 pins, four of which connect to the CPU data bus.

\subsection{Audio Processing Unit (APU)}
The APU is made up of five output channels. These channels include:
	\begin{itemize}
    	\item Two (2) square wave generators
        \item One (1) triangle wave generator
        \item One (1) sample generator
        \item One (1) noise generator
    \end{itemize}
    Each of these channels are fed into their own DAC. Each channel is then combined in the APU Mixer. The APU Mixer is an analog circuit, but can be approximated easily with digital logic.\cite{nesdev}


\section{Schedule and Task Breakdown}
% Presents a detailed breakdown of the schedule into weekly tasks and first and second milestones with measurable goals.

Task Distribution Among Team Members:
\begin{itemize}
\item Cody Anderson is in charge of overlooking the task of CPU creation \\
\item Morgan Skrabut is in charge of overlooking the task of PPU creation \\
\item Ryan Price is in charge of the cartridge and controller connectivity as well as the APU \\
\item Ben Nollan is in charge of designing module framework and all testing and verification
\end{itemize}
First Half Semester Milestones (Week 7):
\begin{itemize}
\item Implement twenty-five opcodes from Super Mario Bros. on the CPU \\
\item PPU Video Output of an image from Super Mario Bros. \\
\item Cartridge to FPGA connectivity \\
\end{itemize}
Second Half Semester Milestones (Week 12):
\begin{itemize}
\item Implement all CPU opcodes required to run Super Mario Bros. \\
\item Finished basic implementation of the PPU \\
\item Complete controller and cartridge connectivity \\
\end{itemize}
\begin{figure}[H]
\centering
\includegraphics[width=\textwidth]{Schedule.PNG}
\caption{\label{fig:Schedule}Project task breakdown for this semester.}
\end{figure}

\section{Bill of Materials}
\begin{figure}[H]
\centering
\includegraphics[width=\textwidth]{bom.png}
\caption{\label{fig:Table}Project Bill of Materials}
\end{figure}

%\section{Testing and Design Verification}
%Present data and/or observations showing how you verified the functionality of your device. Include graphs, tables, diagrams, etc. The data should be presented objectively. Save any discussion for the Discussion section.

%\section{Discussion}
%Discuss implications of the design verification data for your engineering objective. The discussion should address questions or goals presented in your introduction. 

%\section{Conclusions and Future Work}
%Summarize your results and briefly describe future studies that could continue or improve this work.  

% \section{Acknowledgements}
% %Briefly thank people who helped you work on this. DO NOT INCLUDE REFERENCES HERE. 
% 	A special thanks to:
%     \begin{itemize}
%     	\item Jeremy Thomas
%         \item Lukas VanGinneken
%         \item Nick Rivera
%         \item 
        
%     \end{itemize}

% \section{Author Contributions}
% %Include a statement of responsibility that specifies the contribution of every team member. Examples of published "author contributions" statements can be seen at http://blogs.nature.com/nautilus/2007/11/post_12.html 
% 	\begin{itemize}
%     	\item Ben's Acts:
%         \begin{itemize}
%         	\item HDMI Input
%             \item Monitor Identification
%         	\item TMDS Decoding
%         	\item TMDS Encoding
%         	\item HDMI Output
%             \item RGB to HSV
%             \item HSV to RGB
%         	\item Division Pipelining
%         \end{itemize}
        
%     	\item Cody's Acts:
%         \begin{itemize}
%         	\item HSV Transformation Prototyping
%         	\item XYZ Transformation Prototyping
%             \item Colorblindness Simulation Prototyping
%             \item Colorblindness Correction Prototyping
%             \item RGB to HSV
%             \item HSV to RGB
            
%         \item All authors contributed equally to this paper.
%         \end{itemize}
% 	\end{itemize}


% You must cite all references that you significantly consulted for the writing of the proposal in IEEE format in the bibliography. In addition, for all quoted material the source must be referenced. This includes figures, charts and tables that were obtained from other papers.

%\cite{ganyer}
%\cite{nesdev}
%\cite{strothe}

\bibliographystyle{IEEEtran}
\bibliography{bibliography.bib}

\end{document}